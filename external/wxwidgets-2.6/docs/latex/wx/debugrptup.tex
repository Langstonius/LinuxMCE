%%%%%%%%%%%%%%%%%%%%%%%%%%%%%%%%%%%%%%%%%%%%%%%%%%%%%%%%%%%%%%%%%%%%%%%%%%%%%%%
%% Name:        debugrptup.tex
%% Purpose:     wxDebugReportUpload documentation
%% Author:      Vadim Zeitlin
%% Modified by:
%% Created:     2005-03-21
%% RCS-ID:      $Id: debugrptup.tex,v 1.1 2005/03/21 18:28:27 VZ Exp $
%% Copyright:   (c) Vadim Zeitlin 2005
%% License:     wxWindows license
%%%%%%%%%%%%%%%%%%%%%%%%%%%%%%%%%%%%%%%%%%%%%%%%%%%%%%%%%%%%%%%%%%%%%%%%%%%%%%%

\section{\class{wxDebugReportUpload}}\label{wxdebugreportupload}

This class is used to upload a compressed file using HTTP POST request. As this
class derives from wxDebugReportCompress, before upload the report is
compressed in a single .ZIP file.

\wxheading{Derived from}

\helpref{wxDebugReportCompress}{wxdebugreportcompress}

\wxheading{Include files}

<wx/debugrpt.h>


\latexignore{\rtfignore{\wxheading{Members}}}

\membersection{wxDebugReportUpload::wxDebugReportUpload}\label{wxdebugreportuploadwxdebugreportupload}

\func{}{wxDebugReportUpload}{\param{const wxString\& }{url}, \param{const wxString\& }{input}, \param{const wxString\& }{action}, \param{const wxString\& }{curl = \_T("curl")}}

This class will upload the compressed file created by its base class to an HTML
multipart/form-data form at the specified address. The \arg{url} is the upload
page address, \arg{input} is the name of the \texttt{"type=file"} control on
the form used for the file name and \arg{action} is the value of the form
action field. The report is uploaded using \texttt{\textit{curl}} program which
should be available, the \arg{curl} parameter may be used to specify the full
path to it.


\membersection{wxDebugReportUpload::OnServerReply}\label{wxdebugreportuploadonserverreply}

\func{bool}{OnServerReply}{\param{const wxArrayString\& }{WXUNUSED(reply)}}

This function may be overridden in a derived class to show the output from
curl: this may be an HTML page or anything else that the server returned.
Value returned by this function becomes the return value of 
\helpref{wxDebugReport::Process()}{wxdebugreportprocess}.


